\documentclass{article}
%letterpaper,10pt,twoside,onecolumn,titlepage,openright,final

\usepackage{xeCJK}
%language support CJK

%\usepackage[OT2,T1]{fontenc}
\usepackage[french,russian,greek,english]{babel}
\usepackage{geometry}
%packageometry save us from calculate parameters that affect each other
\geometry{inner=1.5cm,outer=1.5cm,top=2cm,bottom=2cm}

%out margin in (ams)book class
\usepackage{graphicx}
%insert external pictures
%usage \includegraphicx[]{}
%parameter in [] : width, height, = ?cm scale = 0.6
\usepackage[fulladjust]{marginnote}
%\setlength{\marginparsep}{5mm}
%\setlength{\marginparwidth}{1in}
%\marginnote instead of \marginpar
\usepackage{xcolor}
\definecolor{hekishoku}{RGB}{0,127,137}
\definecolor{soga}{RGB}{243,244,127}
\definecolor{usukihada}{RGB}{249,241,192}
\definecolor{tokiiro}{RGB}{243,166,150}
\definecolor{codegray}{rgb}{0.5,0.5,0.5}
\definecolor{codepurple}{rgb}{0.58,0,0.82}
\definecolor{mizu}{RGB}{127,204,227}
\definecolor{ao}{RGB}{0,149,217}
\definecolor{kohaku}{RGB}{234,147,10}
\definecolor{daidai}{RGB}{238,120,0}

\usepackage{hyperref}
\hypersetup{
    colorlinks,
    linkcolor=blue,
    citecolor=blue,
    urlcolor=blue
}

\usepackage{amsmath}
\usepackage{amssymb}
%amsmath and ntheorem should be loaded before {fontspec}
%\usepackage{mathtools}
%offer pmatirx*
\usepackage{ulem}
%\uline{}, \uuline{}, \uwave{},\sout{},b\xout{},\dashuline{},\dotuline{},\sout{}(words only)
\usepackage{cancel}
%\cancel{} can be used in formulae!\bcancel{}\xcancle{}\cancelto{<value>}{<expression>}
\usepackage{extarrows}

\usepackage{fontspec,unicode-math}
%unicode-math should be loaded after any othermaths or font-related package in case it needs to overwrite their definitions.
\setmonofont{Courier New}%英文等宽字体
\setsansfont{Noto Sans} %设置无衬线英文字体
\setmainfont{Noto Serif}%{Latin Modern Roman} %设置衬线英文字体,
\setmathfont{STIX Two Math}
\setmathfont{STIX Two Math}[
	range={scr,bfscr},StylisticSet=01
	]
\xeCJKsetup{AutoFakeSlant={true}}
\setCJKmainfont[ItalicFont={FZNewKai-Z03}, BoldFont={Source Han Serif Bold}]{Source Han Serif}  %中文字体设置
\setCJKsansfont{Hiragino Sans GB} %设置中文无衬线字体
\setCJKmonofont{HanaMinA} %设置中文等宽字体字体



\usepackage{listings}
\lstdefinestyle{mystyle_py}{
	backgroundcolor=\color{usukihada},
	commentstyle=\color{hekishoku},
	keywordstyle=\color{ao},
	numberstyle=\tiny\color{codegray},
	stringstyle=\color{daidai}\ttfamily,
	basicstyle=\scriptsize\ttfamily,
	extendedchars=true,
	breakatwhitespace=false,
	breaklines=true,
	captionpos=b,
	keepspaces=true,
	frame=lines,
	numbers=left,
	numbersep=5pt,
	showspaces=false,
	showstringspaces=false,
	showtabs=false,
  	tabsize=4,
	emph=[1]{False, None, True, and, as, assert, async, await, break, class, continue, def, del, elif, else, except, finally, for, from, global, if, import, in, is, lambda, nonlocal, not, or, pass, raise, return, try, while, with, yield},
	emphstyle = [1]\color{ao},
	% Built-ins
	emph=[2]{abs,all,any,basestring,bin,bool,bytearray,callable,chr,classmethod,cmp,compile,complex,delattr,dict,dir,divmod,enumerate,eval,execfile,file,filter,float,format,frozenset,getattr,globals,hasattr,hash,help,hex,id,input,int,isinstance,issubclass,iter,len,list,locals,long,map,max,memoryview,min,next,object,oct,open,ord,pow,property,range,raw_input,reduce,reload,repr,reversed,round,set,setattr,slice,sorted,staticmethod,str,sum,super,tuple,type,unichr,unicode,vars,xrange,zip,apply,buffer,coerce,intern},
	emphstyle = [2]\color{ao}
}

\lstdefinestyle{mystyle_cpp}{
	backgroundcolor=\color{OCM_08},
	commentstyle=\color{OCM_01},
	keywordstyle=\color{OCM_04},	
	numberstyle=\tiny\color{codegray},
	stringstyle=\color{OCM_03}\ttfamily,
	basicstyle=\color{OCM_07}\scriptsize\ttfamily,
	extendedchars=true,
	breakatwhitespace=false,
	breaklines=true,
	captionpos=b,
	keepspaces=true,
	frame=lines,
	numbers=left,
	numbersep=5pt,
	showspaces=false,
	showstringspaces=false,
	showtabs=false,
  	tabsize=4,
	emph=[2]{alignas, alignof, and, and_eq, asm, atomic_cancel, atomic_commit,atomic_noexcept, auto,
	bitand, bitor, bool, break,
	case, catch, char, char8_t, char16_t, char32_t, class, compl, concept, const, consteval, constexpr, constinit, const_cast, continue, co_await, co_return, co_yield,
	decltype, default, delete, do, double, dynamic_cast,
	else, enum, explicit, export, extern,
	false, float, for, friend,
	goto,
	if, inline, int, long,
	mutable, namespace, new, noexcept, not, not_eq, nullptr,
	operator, or, or_eq, private, protected, public,
	reflexpr, register, reinterpret_cast, requires, return,
	short, signed, sizeof, static, static_assert, static_cast, struct, switch,synchronized,
	template, this, thread_local, throw, true, try, typedef, typeid, typename,
	union, unsigned, using, virtual, void, volatile,
	wchar_t, while, xor, xor_eq},
	emphstyle = [2]\color{OCM_02}
	empi=[3]{
		if, elif, else, endif, ifdef, ifndef,
		define, undef, include, line,
		error, pragma, defined,
		__has_include, __has_cpp_attribute,
		export, import, module
	},
	emphstyle = [3]\color{OCM_04}
}

\def\inline{\lstinline[style=mystyle_py]}
\usepackage{tcolorbox}
\tcbuselibrary{listings,skins,breakable,fitting,raster,xparse,external}
\tcbset{
	colback=red!5!white,
	colframe=red!75!black
	}
\newtcolorbox[auto counter,number within=chapter,list inside=columns]{special_columns}[2][]{
	colback=yellow!2!red!2!white,
	colframe=magenta!15!orange!75!white,
	fonttitle=\bfseries,breakable,enhanced,
	title=专栏: #2,#1}

\newcommand{\listofcolumns}{%
  \tcblistof[\chapter*]{columns}{List of Column Boxes}
}
\usepackage{multicol,multirow,booktabs,makecell,array,colortbl}
\usepackage{varwidth}
\usepackage{titlesec,titletoc,etoc}
\usepackage{caption,subcaption}
\usepackage{enumitem}
\setlist{noitemsep}
\setlist[itemize,enumerate,1]{labelindent=\parindent,leftmargin=*}
\setlist[enumerate,1]{label=\fbox{\arabic*.}}

\usepackage{tikz,pgf,pgfplots,pgfplotstable}
\usetikzlibrary{shapes, arrows.meta, trees, chains, positioning, shadows, calc,intersections,math,babel,cd}
\usepgfplotslibrary{external, fillbetween} 

\pgfplotsset{compat=newest}
\usepackage{pdfpages}
\usepackage[linesnumbered,lined,ruled,commentsnumbered]{algorithm2e}

\usepackage{csquotes}
\usepackage[backend=biber]{biblatex}
%\addbibresource{math_bib_file.bib}



\usepackage{amsthm,thmtools}
\let\proof\relax
\let\endproof\relax
\def\thmbegin{\raisebox{0.35ex}{$\scriptstyle\blacktriangleright$}\hspace{0.3em}}
\def\defbegin{\raisebox{0.2ex}{$\displaystyle\maltese$}\hspace{0.3em}}
\declaretheoremstyle[
spaceabove=6pt, spacebelow=6pt,
headfont=\normalfont\bfseries,
notefont=\mdseries, notebraces={(}{)},
headpunct ={:},
headformat=\thmbegin\NAME~\NUMBER \NOTE,
bodyfont=\normalfont\itshape,
postheadspace=1em,
]{general_thm_style}
\declaretheoremstyle[
spaceabove=6pt, spacebelow=6pt,
headfont=\normalfont\bfseries,
notefont=\mdseries, notebraces={(}{)},
headpunct ={:},
headformat=\defbegin\NAME~\NUMBER \NOTE,
bodyfont=\normalfont\itshape,
postheadspace=1em,
]{def_style}
\declaretheorem[numberwithin=section, name=Theorem, style=general_thm_style]{theorem}
\declaretheorem[sibling=theorem, name=Proposition, style=general_thm_style]{proposition}
\declaretheorem[sibling=theorem, name=Definition, style=def_style]{definition}
\declaretheorem[sibling=theorem, name=Lemma, style=general_thm_style]{lemma}
\declaretheorem[sibling=theorem, name=Corollary, style=general_thm_style]{corollary}
\declaretheorem[numberwithin=section, name=e.g.]{example}
\declaretheorem[numberwithin=section, name=e.x.]{exercise}
\declaretheorem[numberwithin=section, name=Remark]{remark}

\declaretheoremstyle[
spaceabove=6pt, spacebelow=6pt,
headfont=\normalfont\itshape,
notefont=\mdseries, notebraces={(}{)},
bodyfont=\normalfont,
postheadspace=1em,
headpunct={:},
headformat=\underline{\NAME \NOTE},
qed=\qedsymbol
]{proof_style}
\declaretheorem[numbered=no, name=Proof, style=proof_style]{proof}

\declaretheoremstyle[
spaceabove=6pt, spacebelow=6pt,
headfont=\normalfont\itshape,
notefont=\mdseries, notebraces={(}{)},
bodyfont=\normalfont,
postheadspace=1em,
headpunct={:},
headformat=\underline{\NAME \NOTE},
qed=$\blacksquare$
]{solution_style}
\declaretheorem[numbered=no, name=Solution, style=solution_style]{solution}



\numberwithin{equation}{section}
\newcommand{\Res}{\mathrm{Res\mathop{}}}
\newcommand{\NaN}{\texttt{NaN}}
\newcommand{\Card}{\mathrm{Card\mathop{}}}
\newcommand{\Rank}{\mathrm{Rank\mathop{}}}
\newcommand{\Var}{\operatorname{Var\mathop{}}}
\newcommand{\Cov}{\operatorname{Cov\mathop{}}}
\newcommand{\Div}{\operatorname{Div\mathop{}}}
\newcommand{\Hom}{\mathbf{Hom}}
\newcommand{\Iso}{\mathbf{Iso}}
\newcommand{\Aut}{\mathbf{Aut}}
\newcommand{\End}{\mathbf{End}}
\newcommand{\Ker}{\mathrm{Ker\mathop{}}}
\newcommand{\Img}{\mathrm{Im\mathop{}}}
\newcommand{\Intr}{\mathrm{Int\mathop{}}}
\newcommand{\abs}[1]{\left\vert #1 \right\vert}
%Absolute value notation
\newcommand{\norm}[1]{\left\Vert #1 \right\Vert}
%Norm notation
\newcommand{\diff}{\mathop{}\!\textrm{d}}

\newcommand{\marginnoteteal}[1]{\marginnote{\textcolor{teal}{#1}}}

\newcommand\doverline[1]{%
		\tikz[baseline=(nodeAnchor.base)]{
			\node[inner sep=0] (nodeAnchor) {$#1$}; 
			\draw[line width=0.1ex,line cap=round] 
				($(nodeAnchor.north west)+(0.0em,0.2ex)$) 
					--
				($(nodeAnchor.north east)+(0.0em,0.2ex)$) 
				($(nodeAnchor.north west)+(0.0em,0.5ex)$) 
					--
				($(nodeAnchor.north east)+(0.0em,0.5ex)$) 
			;
		}
	}

%indispensable!
\title{Peer-graded Assignment \#1}
%\author{\href{mailto:derrring@gmail.com}{Jiongyi Wang}}
\date{Last Updated on: \today}


%    \thanks will become a 1st page footnote.

\begin{document}
\maketitle
\tableofcontents
\section{Problem 1, sets in the complex plane}

\begin{enumerate}
	\item 	$\Gamma_1 = \{z\in\mathbb{C}: \abs{z-3-2i}\leq 1\}$
		\begin{solution}
			$\Gamma_1 = \overline{B(x,1)}$ is the closed ball with radius $1$ centered at $x = 3+2i$ on the complex plane $\mathbb{C}$. Just noticed the boundray $\partial \Gamma_1\subset\Gamma_1$.
		\end{solution}
	\item	 $\Gamma_2 = \{z\in\mathbb{C}: \textrm{Im } z = 2\}$
		\begin{solution}
			There is no constraint on the real part of $z$, however the only value allowed in the imaginary part is $2$, $\Gamma_2$ correspond to a line in complex plane.
		\end{solution}
	\item 	$\Gamma_3=\{z\in \mathbb{C}\setminus \{0\} : 0< \textrm{arg } z < \pi/6\}$
		\begin{solution}
			 Use the fact that $\{z:\textrm{Arg }z = \theta\}$ is a line passing $(0,0)$ and with an angle $\theta$ between (with respect to the positive direction of) real axis. Noticed that $z\in \mathbb{C}\setminus \{0\}$ indiactes that $(0,0)$ is excluded on the graph. $\Gamma_3$ is open in $\mathbb{C}$ hence the boundray is not attained.
		\end{solution}
	\item 	$\Gamma_4=\{z\in\mathbb{C}: \abs{z-1}<\abs{z}\}$
		\begin{solution}
			Let $z=x+iy$ s.t $x,y \in \mathbb{R}$. By squaring two sides, 
			\begin{equation}
				\begin{split}
					\abs{z-1}^2\leq\abs{z}^2 &\implies (x-1)^2+y^2 \leq x^2+y^2\\
					{}&\implies x-\frac{1}{2}\geq 0\\
				\end{split}
			\end{equation}
		that means $\Gamma_4$ indiactes the hypograph of function $x=\frac{1}{2}$. It's also open under the usual topology of $\mathbb{C}$.
		\end{solution}
\end{enumerate}

We could hereby draw all 4 graphs together in the Figure \ref{fig:1}.
\begin{figure}[htbp]
	\centering
	\begin{subfigure}[b]{.4\linewidth}
		\centering
		\begin{tikzpicture}
			\draw[-{Stealth}] (-2,0)--(4,0) node[right]{$1$};
			\draw[-{Stealth}] (0,-2)--(0,3) node[above,left]{$i$};
			\coordinate (center) at (3,2);
			\draw[color = blue,fill = blue!20] (center) circle[radius = 1]; 
			%\path[fill = blue!20] (center) circle[radius = 1];
			\draw[dashed] (3,0) node[below]{$3$}--(center) --(0,2)node[left]{$2$};
			\fill[black,above] (center) circle[radius=1pt];
			\draw[dashed] (3,2)--++(60:1) node[pos=0.5,sloped,auto] {$r=1$};
	  \end{tikzpicture}
	  \caption{$\Gamma_1=\{z\in\mathbb{C}: \abs{z-3-2i}\leq 1\}$}
	\end{subfigure}
	\hfill
	\begin{subfigure}[b]{.4\linewidth}
		\centering
		\begin{tikzpicture}
			\draw[-{Stealth}] (-3,0)--(3,0) node[right]{$1$};
			\draw[-{Stealth}] (0,-2)--(0,3) node[above,left]{$i$};
			\draw[color = blue!] (-3,2)  -- (3,2);
			\draw (0,2) circle[radius = 1pt];
			\node[below left]  at (0,2) {$2$};
	  \end{tikzpicture}
	  \caption{$\Gamma_2 = \{z\in\mathbb{C}: \textrm{Im } z = 2\}$}
	\end{subfigure}

	\vfill
	\begin{subfigure}[b]{.4\linewidth}
		\centering
		\begin{tikzpicture}
			\begin{axis}[
				anchor=origin,at={(0pt,0pt)},
				x=1cm, y=1cm,
				xtick = \empty, ytick = \empty,
				xmin = -3, xmax = 3, ymin=-2, ymax=3,
				axis x line = middle,
				axis y line = middle,
				xlabel = $1$, ylabel=$i$,
			]
			\addplot[dashed, draw=blue, fill=blue!20,opacity=0.5](0,0)--(0:5)arc[start angle=0,end angle=30,radius=5] --cycle;
			\end{axis}
			\draw (0:1)arc[start angle=0,end angle=30,radius=1]node[font=\tiny,pos=0.5,sloped,auto] {$\frac{\pi}{6}$};
			%\fill[blue!](0,0)--(0:3)arc[start angle=0,end angle=30,radius=3]--cycle;
			\draw[blue!20, fill=white] (0,0) circle[radius = 2pt];

			
		\end{tikzpicture}
		\caption{$\Gamma_3=\{z\in \mathbb{C}\setminus \{0\} : 0< \textrm{arg } z < \pi/6\}$}
	\end{subfigure}
	\hfill
	\begin{subfigure}[b]{.4\linewidth}
		\centering
		\begin{tikzpicture}
			\begin{axis}[
				anchor=origin,at={(0pt,0pt)},
				x=1cm, y=1cm,
				xtick = \empty, ytick = \empty,
				xmin = -3, xmax = 3, ymin=-2, ymax=3,
				axis x line = middle,
				axis y line = middle,
				xlabel = $1$, ylabel=$i$,
			]
				\addplot[color=blue, dashed ,mark=\empty, fill=blue!20, opacity=0.5, name path = vert_1] coordinates{(0.5,-3) (0.5,3)};
				\addplot[mark=\empty,  name path = vert_2] coordinates{(4,-3) (4,3)};
				\addplot[mark=\empty, fill=blue!20, opacity=0.5] fill between[of = vert_1 and vert_2];
			\end{axis}
			\draw (0.5,0)circle[radius=1pt] node[below] {$\frac{1}{2}$};

		\end{tikzpicture}
		\caption{$\Gamma_4=\{z\in\mathbb{C}: \abs{z-1}<\abs{z}\}$}
	\end{subfigure}
\caption{The four graphs in Problem 1}
\label{fig:1}
\end{figure}


\section[Problem 2, The nth roots of unity]{Problem 2, The $n$th roots of unity}
By definithon, the $n$th roots of a number $z$ is $w$ that satisfy the equation:
\begin{equation}\label{eq:complex-roots}
	w^n =re^{i\theta},\quad \theta = \textrm{Arg }z \in [-\pi,\pi)
\end{equation}
simply noticed that we also require that $\mathrm{Arg }w\in [-\pi,\pi)$:
\begin{equation}\label{eq:complex-roots-set}
	w =\left\{ \sqrt[n]{r}e^{\frac{i\theta}{n}+\frac{2k\pi}{n}}: k\in\{0,1,\ldots, n-1\}\right\}
\end{equation}
in our case to find $n$th unit roots, it equals to solve equation \ref{eq:complex-roots} with $r=1, \theta=0$, and our desired results can be directly inferred from formula \ref{eq:complex-roots-set}:

\begin{equation}
	U_n =\left\{e^{\frac{2k\pi}{n}}: k\in\{0,1,\ldots,n-1\}\right\}
\end{equation}

it's quite obvious that $\#U_n = n$, thus all $6$th roots of unity are enumerated in $U_6$, which is
\begin{equation}
	U_6 = \left\{e^0, e^{\frac{i\pi}{3}} , e^{\frac{2i\pi}{3}},e^{i\pi},e^{\frac{4i\pi}{3}},e^{\frac{5i\pi}{3}}\right\}
\end{equation}

\subsection{How to draw roots of unit}
Although the polar coordinate of complex number $z = x+iy$ can be calculate with the help of $\cos\theta+i\sin\theta =e^{i\theta}$, it's a little bit cumbersome to decide explicitly the value of $\theta$:
\begin{equation}
	\theta = \left\{
	\begin{array}{lc}
		\arccos\frac{x}{r}& y\geq0, r\neq 0\\
		-\arccos\frac{x}{r}& y<0\\
		NaN& r=0
	\end{array}\right\vert \quad r = \sqrt{x^2+y^2}
\end{equation}

Espacially when the value of $\frac{x}{r}$ is not such special as $1,\frac{1}{2},\frac{\sqrt{3}}{2},\frac{\sqrt{2}}{2}$, etc. Intutively $U_6$ is cyclotomic, so its elements are uniformly distributed over the unit circle, that's sufficient for us to locate their position. To get a little bit more algebraic (and rigorous), $U_n$ is essentially a cyclic group of order $n$, one of its generators is $e^{\frac{i\pi}{n}}$(there are many generators in the same cyclic group, in our case any $e^{\frac{ik\pi}{3}}$ with $\gcd(k,6)=1$ is a generator, our construction can be started from any generator). Thus, when we draw $e^{\frac{ik\pi}{3}}$, it means that we rotate counterclockwise the generator $e^{\frac{1\pi}{3}}$ by $k$ times. We show this process in Figure \ref{fig:2}.

\begin{figure}

	\begin{subfigure}[t]{.3\linewidth}
		\centering
		\begin{tikzpicture}
			\draw[-{Stealth}] (-1.5,0)--(1.5,0) node[right]{$1$};
			\draw[-{Stealth}] (0,-1.5)--(0,1.5) node[above,left]{$i$};
			\draw[name path = unit_circle] (0,0)circle[radius=1];
			\draw[dashed, name path = root_1] (0,0) -- (canvas polar cs:radius=1cm,angle=60);
			\fill[ name intersections ={of = root_1 and unit_circle} ] [red , opacity=0.5, every node/.style={black, opacity=1}]
			(intersection-1)circle[radius=1pt] node[above right] {$e^{i\frac{\pi}{3}}$};
		\end{tikzpicture}
		\caption{pick and draw the generator $e^{\frac{i\pi}{3}}$}
	\end{subfigure}
	\hfill
	\begin{subfigure}[t]{.3\linewidth}
		\centering
		\begin{tikzpicture}
			\draw[-{Stealth}] (-1.5,0)--(1.5,0) node[right]{$1$};
			\draw[-{Stealth}] (0,-1.5)--(0,1.5) node[above,left]{$i$};
			\draw[name path = unit_circle]  (0,0)circle[radius=1];
			\foreach \x in {1,2,3}{	
					\draw[dashed, name path/.expanded = root_\x] (0,0) -- (canvas polar cs:radius=1cm,angle=60*\x);
					\draw[->,red] (60*\x:1.2)arc[start angle=60*\x,end angle=60*\x+30,radius=1.2]node[font=\tiny,pos=0.5,rotate=60*(\x-1),above] {$\frac{\pi}{3}$};
					\fill[name intersections ={of = {root_\x} and unit_circle}][red , opacity=0.5, every node/.style={black, opacity=1}]
					(intersection-1) circle[radius=1pt]node {$e^{i\frac{{\x}\pi}{3}}$};
				}
				\draw[->,red!50] (220:1.3)arc[start angle=220,end angle=300,radius=1.3]node {};
		\end{tikzpicture}
		\caption{rotate counterclockwise with angle $\frac{\pi}{3}$ and repeat }
	\end{subfigure}
	\hfill
	\begin{subfigure}[t]{.3\linewidth}
		\centering
		\begin{tikzpicture}
			\draw[-{Stealth}] (-1.5,0)--(1.5,0) node[right]{$1$};
			\draw[-{Stealth}] (0,-1.5)--(0,1.5) node[above,left]{$i$};
			\draw[name path global= unit_circle] (0,0)circle[radius=1];
			\foreach \x in {0,1,2,3,4,5}{
				\draw[dashed, name path global/.expanded=root_\x] (0,0) -- (canvas polar cs:radius=1cm,angle=60*\x);
				\fill[name intersections ={of = {root_\x} and unit_circle}][red , opacity=0.5, every node/.style={black, opacity=1,label distance=0.1cm}]
				(intersection-1) circle[radius=1pt]node[shift=(60*\x:0.4)]{$e^{i\frac{{\x}\pi}{3}}$};
			}
		\end{tikzpicture}
		\caption{all $6$th roots of unity}
	\end{subfigure}
	\caption{A feasible process to draw $6$th unit roots}
	\label{fig:2}
\end{figure}





\end{document}
